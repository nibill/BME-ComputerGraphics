% --------------------------------------------------------------
% This is all preamble stuff that you don't have to worry about.
% Head down to where it says "Start here"
% --------------------------------------------------------------

\documentclass[12pt]{article}

\usepackage[margin=1in]{geometry}
\usepackage{amsmath,amsthm,amssymb}
\usepackage{graphicx} %This allows to include eps figures
\usepackage{subcaption}
\usepackage[section]{placeins}
\usepackage{layout}
\usepackage{etoolbox}
\usepackage{mathabx}
% This is to include code
\usepackage{listings}
\usepackage{xcolor}
\definecolor{dkgreen}{rgb}{0,0.6,0}
\definecolor{gray}{rgb}{0.5,0.5,0.5}
\definecolor{mauve}{rgb}{0.58,0,0.82}
\lstdefinestyle{Python}{
    language        = Python,
    basicstyle      = \ttfamily,
    keywordstyle    = \color{blue},
    keywordstyle    = [2] \color{teal}, % just to check that it works
    stringstyle     = \color{green},
    commentstyle    = \color{red}\ttfamily
}

\newcommand{\N}{\mathbb{N}}
\newcommand{\Z}{\mathbb{Z}}

\newenvironment{theorem}[2][Theorem]{\begin{trivlist}
\item[\hskip \labelsep {\bfseries #1}\hskip \labelsep {\bfseries #2.}]}{\end{trivlist}}
\newenvironment{lemma}[2][Lemma]{\begin{trivlist}
\item[\hskip \labelsep {\bfseries #1}\hskip \labelsep {\bfseries #2.}]}{\end{trivlist}}
\newenvironment{exercise}[2][Exercise]{\begin{trivlist}
\item[\hskip \labelsep {\bfseries #1}\hskip \labelsep {\bfseries #2.}]}{\end{trivlist}}
\newenvironment{reflection}[2][Reflection]{\begin{trivlist}
\item[\hskip \labelsep {\bfseries #1}\hskip \labelsep {\bfseries #2.}]}{\end{trivlist}}
\newenvironment{proposition}[2][Proposition]{\begin{trivlist}
\item[\hskip \labelsep {\bfseries #1}\hskip \labelsep {\bfseries #2.}]}{\end{trivlist}}
\newenvironment{corollary}[2][Corollary]{\begin{trivlist}
\item[\hskip \labelsep {\bfseries #1}\hskip \labelsep {\bfseries #2.}]}{\end{trivlist}}

\begin{document}

% --------------------------------------------------------------
%                         Start here
% --------------------------------------------------------------

%\renewcommand{\qedsymbol}{\filledbox}

\title{Assignment 01}%replace X with the appropriate number
\author{Thomas Buchegger, Carolina Duran, Nalet Meinen \\ %replace with your name
Computer Graphics
}

\maketitle

\section{Derivation for the Cylinder Intersections}

The implicit representation of a cylinder goes as
\[ \|(x - c) \times \vec a\|^2 - r^2 = 0 \]

\noindent Where $\vec a$ is the direction of the cylinder from the center 
\newline

\noindent We plug in the ray parametrization form

\[ x = t \cdot \vec d + o \]

\noindent therefore we get

\[ \|(t \cdot d + o - c) \times \vec a\|^2 - r^2 = 0 \]


\noindent resulting in the final from

\begin{align*} t^2 \cdot \|\vec d \times \vec a\|^2 + t\cdot 2(\vec d \times \vec a)(o-c)\times \vec a + \|(o-c)\times \vec a\|^2 -r^2 = 0
\end{align*}

\section{Derivation for the Cylinder Intersection Normal}
The normal which points from the cylinder outwards at point $x$ can be calculated by two consecutive cross products:
 \[ \vec n = \frac{(\vec a \times (x-c))\times \vec a}{\|(\vec a \times (x-c))\times \vec a\|}\]


\end{document}